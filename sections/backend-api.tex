%%%%%%%%%%%%%%%%%%%%%%%%%%%%%%%%%%%%%
% BACKEND-KAPITEL
%%%%%%%%%%%%%%%%%%%%%%%%%%%%%%%%%%%%%
\chapter{Backend-Architektur und API-Design}
\renewcommand{\authorinitials}{MT}
\label{chap:backend}

\section{Backend-Aufbau und Signalverarbeitung}

Das Backend der Smart Shopping App basiert wie bereits erwähnt auf einer modernen Express.js-Architektur mit TypeScript und folgt dem bewährten Controller-Service-Pattern. Die Anwendung ist in mehrere Schichten unterteilt, um eine klare Trennung der Verantwortlichkeiten zu gewährleisten.

\subsection{Express.js App-Struktur}

Die Hauptanwendung wird in \texttt{app.ts} konfiguriert und folgt einer \textbf{Middleware-Pipeline}-Architektur. Diese Pipeline verarbeitet jeden eingehenden Request durch verschiedene Schichten: \textbf{JSON-Parsing} wandelt automatisch JSON-Requests in JavaScript-Objekte um, \textbf{CORS} erlaubt dem Frontend den Zugriff auf die API, und die \textbf{Authentifizierung} überprüft die Benutzeridentität. Anschließend leiten die \textbf{API Routes} zu den entsprechenden Endpunkten weiter, während der \textbf{Error Handler} am Ende alle Fehler abfängt und strukturiert beantwortet. Diese Architektur sorgt für eine saubere Trennung der Verantwortlichkeiten und macht das System wartbar und erweiterbar. Im weiteren Verlauf des Kapitels werden die einzelnen Komponenten wie \textbf{Authentifizierung}, \textbf{Routing-Struktur} und \textbf{Error Handling} detailliert erläutert.

\subsection{Authentifizierung mit Clerk}

Die Authentifizierung erfolgt über Clerk, einem Identity-as-a-Service-Provider. Jeder API-Request muss einen gültigen Bearer-Token im Authorization-Header enthalten:

\begin{lstlisting}[style=typescriptstyle,caption={Authentifizierungs-Middleware}]
export const requireAuth = () => (
  req: express.Request,
  res: express.Response,
  next: express.NextFunction
) => {
  const { userId } = getAuth(req);

  if (!userId) {
    res.status(401).json({
      error: "Unauthorized",
      message: "Authentication required",
    });
    return;
  }

  next();
};
\end{lstlisting}

Die \texttt{getAuth(req)}-Funktion extrahiert automatisch die User-ID aus dem JWT-Token (JSON Web Token), die dann in Controllern für user-spezifische Operationen verwendet wird.

\subsection{Error Handling und Middleware}

Das Backend implementiert ein zentrales Error-Handling-System, das verschiedene Fehlertypen unterschiedlich behandelt:

\begin{lstlisting}[style=typescriptstyle,caption={Zentraler Error Handler}]
export default function errorHandler(
  err: any,
  req: Request,
  res: Response,
  next: NextFunction
): void {
  console.error(err);
  if (typeof err.message === "string" && err.message.startsWith("DB_ERROR")) {
    res.status(500).json({ error: err.message });
    return;
  }
  res.status(err.status || 500).json({ error: err.message || "Unknown Error" });
}
\end{lstlisting}

Datenbankfehler werden speziell behandelt und mit dem Präfix "DB\_ERROR" gekennzeichnet, während andere Fehler mit einem generischen 500-Status beantwortet werden.

\subsection{Request/Response-Flow}

Der typische Request-Flow durchläuft folgende Schichten:

\begin{enumerate}
    \item \textbf{Route Layer:} Definiert HTTP-Endpunkte und HTTP-Methoden
    \item \textbf{Controller Layer:} Extrahiert Request-Daten und delegiert an Services
    \item \textbf{Service Layer:} Implementiert Business Logic und Datenbankoperationen
    \item \textbf{Database Layer:} Prisma ORM für typsichere Datenbankzugriffe
\end{enumerate}

\section{API-Design und Endpunkte}

Die API folgt RESTful-Prinzipien und ist in logische Ressourcen-Gruppen unterteilt. Jeder Endpunkt ist typgesichert und implementiert konsistente Response-Formate.

\subsection{Routing-Struktur}

Die Routen sind \textbf{modular organisiert} und werden über einen zentralen Router verwaltet. Die Routing-Struktur folgt dem bewährten Express.js-Pattern mit separaten Router-Modulen für jede Ressource. \textbf{Products}, \textbf{Shopping Lists} und \textbf{Stores} haben jeweils eigene Dateien für jeden Layer, die über den zentralen API-Router eingebunden werden. Diese Struktur ermöglicht eine klare Trennung der Verantwortlichkeiten und erleichtert die Wartung und Erweiterung der API.

\subsection{Controller-Service-Pattern}

Das Backend implementiert das Controller-Service-Pattern für eine klare Trennung zwischen HTTP-Logik und Business Logic:

\begin{lstlisting}[style=typescriptstyle,caption={Controller-Beispiel: Shopping List}]
export const addItemToShoppingList = async (
  req: Request,
  res: Response,
  next: NextFunction
) => {
  const { userId } = getAuth(req);
  const shoppingList = await service.addItemToShoppingList(
    userId as string,
    Number(req.body.variantId),
    Number(req.body.quantity)
  );
  res.json(shoppingList);
};
\end{lstlisting}

Controller sind für Request/Response-Handling zuständig, während Services die eigentliche Business Logic implementieren.

\subsection{API-Dokumentation}

Alle API-Endpunkte erfordern Authentifizierung über einen Bearer Token im Authorization-Header. Im folgenden Teil werden die API-Requests aufgezählt und kurz beschrieben:

\subsubsection{\texttt{POST /api/shoppinglist/}}

Erstellt eine neue Einkaufsliste für den authentifizierten Benutzer. Es ist kein Request Body erforderlich. Die Antwort enthält die erstellte Einkaufsliste mit ID, Name, User-ID und Erstellungszeitstempel.

\begin{lstlisting}[language=JSON]
{
  "id": 1,
  "name": "Default List",
  "userId": "user_123",
  "createdAt": "2024-01-15T10:30:00Z"
}
\end{lstlisting}

\subsubsection{\texttt{GET /api/shoppinglist/}}

Ruft die aktuelle Einkaufsliste des Benutzers ab. Die Antwort enthält die Einkaufsliste mit allen enthaltenen Items, einschließlich Produktdetails und Markeninformationen.

\begin{lstlisting}[language=JSON]
{
  "id": 1,
  "name": "Default List",
  "items": [
    {
      "id": 1,
      "variantId": 5,
      "quantity": 2,
      "variant": {
        "id": 2,
        "size": 1,
        "unit": "stueck",
        "product": {
          "id": 3,
          "name": "Bio Milch",
          "brandId": 4
        }
      }
    }
  ]
}
\end{lstlisting}

\subsubsection{\texttt{PUT /api/shoppinglist/}}

Fügt ein Produkt zur Einkaufsliste hinzu. Der Request Body muss die Varianten-ID und die gewünschte Menge enthalten. Die Antwort enthält die aktualisierte Einkaufsliste.

\begin{lstlisting}[language=JSON]
{
  "variantId": 5,
  "quantity": 2
}
\end{lstlisting}

\subsubsection{\texttt{DELETE /api/shoppinglist/}}

Entfernt ein Produkt aus der Einkaufsliste. Der Request Body muss die Varianten-ID des zu entfernenden Produkts enthalten.

\begin{lstlisting}[language=JSON]
{
  "variantId": 5
}
\end{lstlisting}

\subsubsection{\texttt{GET /api/shoppinglist/items}}

Ruft alle Einkaufslisten-Items mit detaillierten Produktinformationen ab. Die Antwort enthält eine Liste aller Items mit vollständigen Produktdetails wie Größe, Einheit, Name, Preis, Store-Name, Marken-Name und Bild-URL.

\begin{lstlisting}[language=JSON]
[
  {
    "id": 1,
    "variantId": 5,
    "quantity": 2,
    "size": "1",
    "unit": "l",
    "name": "Bio Milch",
    "price": 1.29,
    "storeName": "Aldi Sued",
    "brandName": "GUT BIO",
    "imageUrl": "https://example.com/milk.jpg"
  }
]
\end{lstlisting}

\subsubsection{\texttt{GET /api/products}}

Ruft Produkte mit optionalen Filtern ab. Als Query-Parameter können Suchbegriff, Kategorie, Store, Limit und Offset für Pagination übergeben werden. Die Antwort enthält eine Liste von Produkten mit Markeninformationen, Varianten und Store-Angeboten sowie Metadaten für Pagination.

\begin{lstlisting}[language=JSON]
{
  "products": [
    {
      "id": 1,
      "name": "Bio Milch",
      "brand": { "name": "GUT BIO" },
      "variants": [
        {
          "id": 5,
          "size": "1",
          "unit": "l",
          "storeOffers": [
            {
              "price": 1.29,
              "store": { "name": "Aldi Sued" }
            }
          ]
        }
      ]
    }
  ],
  "total": 150,
  "limit": 50,
  "offset": 0
}
\end{lstlisting}

\subsubsection{\texttt{GET /api/stores}}

Ruft alle verfügbaren Stores ab. Die Antwort enthält eine Liste aller Stores mit ID, Name und zugehöriger Farbe für die UI-Darstellung.

\begin{lstlisting}[language=JSON]
[
  {
    "id": 1,
    "name": "Aldi Sued",
    "color": "#4B946A"
  },
  {
    "id": 2,
    "name": "Aldi Nord", 
    "color": "#E74C3C"
  }
]
\end{lstlisting}

\section{Datenbankinteraktionen}

Der Aufbau der Datenbank wird im nächsten Kapitel erklärt. Hier werden die Interaktionen mit ihr beschrieben.

\subsection{Prisma ORM und Datenbankzugriff}

Das Backend nutzt Prisma als TypeScript-first ORM für typsichere Datenbankoperationen. Die Datenbankverbindung wird über einen zentralen Client verwaltet. Die Implementierung erfolgt durch den Import des \texttt{PrismaClient} aus dem \texttt{@prisma/client} Paket und die Erstellung einer Instanz mit \texttt{export const db = new PrismaClient()}. Dieser Client wird dann in allen Service-Layer-Komponenten verwendet, um Datenbankoperationen durchzuführen. Der PrismaClient übernimmt automatisch das Connection Pooling und stellt sicher, dass alle Datenbankverbindungen effizient verwaltet werden.

\subsection{Service-Layer und Business Logic}

Services implementieren die Business Logic und abstrahieren komplexe Datenbankoperationen. Ein großer Vorteil von Prisma ist, dass keine SQL-Syntax benötigt wird - die Abfragen sind bereits sehr einfach und intuitiv. Zusätzlich können problemlos andere Tabellen in einem Request integriert werden, was komplexe Joins überflüssig macht.

Der Grundaufbau einer Service-Methode folgt einem klaren Muster: Zuerst erfolgt die Datenbankabfrage mit Prisma, dann die Validierung der Ergebnisse, anschließend die Business Logic (wie Aggregationen oder Transformationen) und schließlich die Rückgabe der aufbereiteten Daten. Ein typisches Beispiel ist der ShoppingListService:

\begin{lstlisting}[style=typescriptstyle,caption={Service-Layer Beispiel}]
export const getShoppingListItems = async (userId: string) => {
  const shoppingList = await db.shoppingList.findFirst({
    where: { userId: userId },
    include: {
      items: {
        include: {
          variant: {
            include: {
              product: { include: { brand: true } },
              storeOffers: { include: { store: true } },
              images: true,
            },
          },
        },
      },
    },
  });

  if (!shoppingList) return null;

  // Business Logic: Aggregation von Duplikaten
  const shopitems = shoppingList.items.reduce((acc, item) => {
    const existingItem = acc.find((i) => i.variantId === item.variantId);
    if (existingItem) {
      existingItem.quantity += item.quantity;
    } else {
      acc.push(item);
    }
    return acc;
  }, []);

  return shopitems.map((item) => ({
    id: item.variant?.product?.id,
    variantId: item.variantId,
    quantity: item.quantity,
    size: item.variant?.size,
    unit: item.variant?.unit,
    name: item.variant?.product?.name,
    price: item.variant?.storeOffers[0]?.price,
    storeName: item.variant?.storeOffers[0]?.store?.name,
    brandName: item.variant?.product?.brand?.name,
    imageUrl: item.variant?.images[0]?.sourceUrl,
  }));
};
\end{lstlisting}

\section{Performance und Skalierung}

Das Backend implementiert bereits einige \textbf{Performance-Optimierungen}. Prisma's \texttt{include}-Funktionalität ermöglicht das schon kurz erwähnte \textbf{Eager Loading}, wodurch verwandte Daten in einer einzigen Abfrage geladen werden können. Dies reduziert die Anzahl der Datenbankzugriffe erheblich. Zusätzlich übernimmt Prisma automatisch das \textbf{Connection Pooling} und verwaltet die Datenbankverbindungen effizient. Die \textbf{stateless-Architektur} des Backends ermöglicht es, dass mehrere Instanzen parallel laufen können, ohne dass Daten zwischen ihnen geteilt werden müssen.

Für zukünftige Skalierung könnten weitere Optimierungen implementiert werden. \textbf{Horizontal Scaling} durch Load Balancing würde es ermöglichen, mehrere Backend-Instanzen hinter einem Load Balancer zu betreiben. Die Datenbank könnte durch \textbf{Read-Replicas} erweitert werden, um Leseoperationen zu verteilen und die Hauptdatenbank zu entlasten. \textbf{Redis-Integration} für Caching würde häufig abgerufene Daten im Speicher halten und die Antwortzeiten deutlich verbessern. Zusätzlich könnten \textbf{Datenbankindizes} für häufig verwendete Abfragen optimiert werden.

\section{Sicherheitsaspekte}

Das Backend implementiert umfassende \textbf{Sicherheitsmaßnahmen} auf mehreren Ebenen. Die \textbf{Authentifizierung} erfolgt über Clerk, der die \textbf{JWT-Token-Validierung} und das \textbf{Session-Management} übernimmt. Jeder API-Request muss einen gültigen Bearer-Token enthalten, aus dem die User-ID extrahiert wird. Diese User-ID wird dann für alle Datenbankoperationen verwendet, wodurch eine strikte \textbf{User-Isolation} gewährleistet ist, Benutzer können nur auf ihre eigenen Daten zugreifen.

Auf technischer Ebene bietet \textbf{TypeScript-Typisierung} bereits zur Kompilierzeit eine erste Validierung der Eingabedaten. Das \textbf{Prisma ORM} verhindert automatisch \textbf{SQL-Injection}-Angriffe durch parametrisierte Abfragen. Die \textbf{CORS-Konfiguration} schützt vor unerwünschten Cross-Origin-Requests. Das zentrale \textbf{Error-Handling} stellt sicher, dass keine sensiblen Informationen in Fehlermeldungen preisgegeben werden. Für zukünftige Erweiterungen könnten \textbf{Rate Limiting} zum Schutz vor DDoS-Angriffen und erzwungene \textbf{HTTPS-Verschlüsselung} implementiert werden.

\section{Deployment}

\subsection{Umgebungskonfiguration}

Das Backend nutzt \textbf{Umgebungsvariablen} für eine flexible und sichere Konfiguration. Diese Variablen werden über \texttt{.env}-Dateien oder direkt in der Deployment-Umgebung gesetzt. Die \textbf{DATABASE\_URL} definiert die Verbindung zur PostgreSQL-Datenbank und wird von Prisma automatisch erkannt. Für die \textbf{Authentifizierung} werden\\ \textbf{CLERK\_SECRET\_KEY} und \textbf{CLERK\_PUBLISHABLE\_KEY} verwendet, die von der Clerk-Plattform bereitgestellt werden. Der \textbf{PORT} bestimmt, auf welchem Port der Server läuft, während \textbf{NODE\_ENV} die Laufzeitumgebung definiert und das Verhalten der Anwendung beeinflusst. Diese Konfiguration ermöglicht es, die gleiche Codebasis in verschiedenen Umgebungen zu verwenden, ohne Änderungen am Code vornehmen zu müssen.

\subsection{Build und Deployment}

Das Backend wird über \textbf{TypeScript-Kompilierung} in JavaScript-Code umgewandelt. Der \textbf{Build-Prozess} mit \texttt{npm run build} kompiliert den TypeScript-Code und erstellt eine optimierte Version im \texttt{dist}-Verzeichnis. Für das \textbf{Deployment} wird der kompilierte Code zusammen mit den Umgebungsvariablen in die Zielumgebung übertragen. Der Server wird dann mit \texttt{npm start} gestartet, der den kompilierten Code aus \texttt{dist/index.js} ausführt. Die \textbf{API-Endpunkte} sind dann über HTTP/HTTPS erreichbar und können von der Frontend-App angesprochen werden. Wichtig ist, dass die Datenbankverbindung und alle Umgebungsvariablen korrekt konfiguriert sind, damit die API ordnungsgemäß funktioniert. Für die \textbf{Skalierung} können mehrere Backend-Instanzen parallel betrieben werden, da das Backend stateless ist und keine lokalen Daten speichert.



\section{Integration Tests}

\subsection{Test-Setup und Framework}

Das Backend verwendet \textbf{Jest} als Test-Framework mit \textbf{Supertest} für HTTP-API-Tests. Die Jest-Konfiguration ist auf \textbf{TypeScript} optimiert und verwendet das \textbf{ts-jest}-Preset für nahtlose Unterstützung. Die \textbf{Node.js}-Testumgebung ermöglicht das Testen von Backend-Code ohne Browser-Simulation. \textbf{Path-Aliase} werden über \texttt{moduleNameMapper} konfiguriert, um die \texttt{@/}-Imports korrekt aufzulösen. Die \texttt{testMatch}-Konfiguration stellt sicher, dass nur Dateien im \texttt{\_\_tests\_\_}-Verzeichnis mit der Endung \texttt{.test.ts} als Tests erkannt werden.

\subsection{Authentifizierungstests}

Die \textbf{Authentifizierung} wird umfassend getestet. Die Tests überprüfen verschiedene Szenarien: Fehlende Authorization-Header führen zu 401-Fehlern, ungültige Tokens werden korrekt abgelehnt und gültige Tokens ermöglichen den Zugriff auf geschützte Endpunkte. Diese Tests stellen sicher, dass die Sicherheitsmaßnahmen ordnungsgemäß funktionieren und unbefugte Zugriffe verhindert werden.

\subsection{Shopping List Integration Tests}

Die Shopping List Funktionalität wird mit umfassenden \textbf{Integration Tests} abgedeckt. Diese Tests simulieren echte API-Aufrufe und überprüfen das gesamte System von der Request-Verarbeitung bis zur Datenbankoperation. Die Tests decken alle wichtigen Szenarien ab: Das Erstellen neuer Einkaufslisten, das Abrufen bestehender Listen, das Hinzufügen von Produkten sowie die Validierung von Eingabedaten. Besonders wichtig sind die Validierungstests, die sicherstellen, dass ungültige Eingaben wie Null-Werte, negative Mengen oder nicht-existierende Produkt-IDs korrekt abgelehnt werden. Die Test verwenden \texttt{'Bearer validToken'} als simulierten Token und führen so eine authentifizierte Benutzeranfragen durch.

\begin{lstlisting}[style=typescriptstyle,caption={Shopping List Integration Test}]
describe('Shopping List API', () => {
  it('should add item to shopping list', async () => {
    const response = await request(app)
      .put('/api/shoppinglist/')
      .set('Authorization', `Bearer validToken`)
      .send({
        variantId: 1,
        quantity: 2
      });
    expect(response.status).toBe(200);
    expect(response.body.items).toHaveLength(1);
    expect(response.body.items[0].quantity).toBe(2);
  });

  it('should reject invalid variant ID', async () => {
    const response = await request(app)
      .put('/api/shoppinglist/')
      .set('Authorization', `Bearer validToken`)
      .send({
        variantId: 99999,
        quantity: 1
      });
    expect(response.status).toBe(400);
  });
});
\end{lstlisting}

\subsection{Mocking-Strategien}

Die Tests verwenden eine Kombination aus \textbf{Mocking} und \textbf{echter Infrastruktur}. Die \textbf{Clerk-Authentifizierung} wird gemockt, um verschiedene Authentifizierungsszenarien zu simulieren und Test-User-IDs bereitzustellen. Die \textbf{echte Datenbank} wird für vollständige Integration Tests verwendet, was eine realistische Testumgebung gewährleistet. Die \textbf{Middleware-Pipeline} wird vollständig durchlaufen, wodurch alle Authentifizierungs- und Validierungsschritte getestet werden. Diese Strategie ermöglicht es, sowohl die Authentifizierungslogik als auch die Datenbankintegration in einem einzigen Test zu überprüfen.

\subsection{Test-Ausführung}

Die Test-Ausführung erfolgt über \texttt{npm test} und nutzt die zuvor beschriebene Jest-Konfiguration. Für die Entwicklung steht ein \textbf{Watch-Modus} zur Verfügung, der Tests automatisch bei Code-Änderungen neu ausführt. Spezifische Test-Dateien können über \texttt{npm test -- filename.test.ts} gezielt ausgeführt werden, was besonders bei der Entwicklung einzelner Features nützlich ist. Die Test-Ausführung erfolgt in der \textbf{Node.js}-Umgebung und unterstützt sowohl einzelne Test-Suites als auch die vollständige Test-Suite. Diese Flexibilität ermöglicht eine effiziente Entwicklung und Debugging der Backend-Funktionalität.

\subsection{Ausblick: Erweiterte Tests}

Basierend auf der beschriebenen Test-Strategie fokussiert sich das Backend bewusst auf \textbf{Integration Tests} und verzichtet auf \textbf{Unit Tests}. Diese Entscheidung liegt daran, dass die meisten Backend-Funktionen stark mit der Datenbank und Authentifizierung verknüpft sind. Unit Tests würden umfangreiches Mocking der Datenbank und Authentifizierung erfordern, was die Tests komplexer und weniger aussagekräftig machen würde. Die Integration Tests decken bereits die wichtigsten Szenarien ab und bieten eine realistische Testumgebung. Für zukünftige Erweiterungen könnten \textbf{Performance-Tests} für kritische Endpunkte und \textbf{Load-Tests} für die Skalierbarkeit implementiert werden, um die Robustheit der Anwendung weiter zu verbessern.

\section{Fazit}

Das Backend der Smart Shopping App implementiert eine moderne, skalierbare Architektur mit Express.js, TypeScript und Prisma. Die API folgt RESTful-Prinzipien und bietet eine vollständige Authentifizierung über Clerk. Durch das Controller-Service-Pattern wird eine klare Trennung der Verantwortlichkeiten gewährleistet, während Integration Tests die Ausführbarkeit sicherstellen.

Die Architektur ist darauf ausgelegt, zukünftige Erweiterungen wie Caching, Rate Limiting und erweiterte Sicherheitsfeatures einfach zu integrieren.

\cleardoublepage

