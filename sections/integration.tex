%%%%%%%%%%%%%%%%%%%%%%%%%%%%%%%%%%%%%
% INTEGRATION VON FRONTEND UND BACKEND
%%%%%%%%%%%%%%%%%%%%%%%%%%%%%%%%%%%%%
\chapter{Integration von Frontend und Backend}
\renewcommand{\authorinitials}{DH}

\label{chap:integration}

\section{Überblick der Client–Server-Kommunikation}
Die Kommunikation zwischen dem mobilen Frontend und dem Express‑Backend erfolgt über eine REST‑API, die im Frontend durch einen zentralen \texttt{ApiService} gekapselt wird. Dieser nutzt Axios, um HTTP‑Requests an das Backend zu senden und Authentifizierungs­token automatisch in die Header einzufügen.

\subsection{ApiService im Frontend}
Im Frontend definiert \texttt{api.ts} die \texttt{ApiService}-Klasse mit Methoden für alle CRUD‑Operationen der Einkaufslisten. Ein exemplarischer Ausschnitt:

\begin{lstlisting}[language=TypeScript,caption={Definition der \texttt{createShoppingList}-Methode im \texttt{ApiService}}]
public async createShoppingList(): Promise<ShoppingListItem[]> {
const response = await this.axiosInstance.post("/shoppinglist/");
return response.data || [];
}
\end{lstlisting}

In der React‑Komponente \texttt{ListScreen.tsx} wird diese Methode wie folgt genutzt:

\begin{lstlisting}[language=TypeScript,caption={Aufruf von \texttt{createShoppingList} im Frontend}]
const handleCreateShoppingList = async () => {
setIsCreatingList(true);
try {
const apiResponse = await api.createShoppingList();
setShoppingList(apiResponse);
setHasShoppingList(true);
Alert.alert("Success", "Shopping list created successfully!");
} catch {
Alert.alert("Error", "Could not create shopping list");
} finally {
setIsCreatingList(false);
}
};
\end{lstlisting}

\section{Backend‑Implementierung der Endpunkte}
Das Backend exponiert unter \texttt{/api/shoppinglist} eine Reihe von Endpunkten, die in \texttt{shoppingLists.ts} (Router) und \texttt{shoppingListsController.ts} (Controller) definiert sind.

\subsection{Router‑Definition}
Der Einkaufslisten‑Router hierzu:

\begin{lstlisting}[language=TypeScript,caption={Routing des Einkaufslisten‑Endpoints im Backend (\texttt{shoppingLists.ts})}]
import { Router } from "express";
import { createShoppingList, getShoppingListItems } from "@/controllers/shoppingListsController";
import { asyncHandler } from "@/utils/asyncHandler";

const router = Router();

// POST /api/shoppinglist/ -> createShoppingList
router.post("/", asyncHandler(createShoppingList));

// GET /api/shoppinglist/items -> getShoppingListItems
router.get("/items", asyncHandler(getShoppingListItems));

export default router;
\end{lstlisting}

\subsection{Controller‑Logik}
Im Controller werden die Anfragen authentifiziert und an den Service weitergeleitet:

\begin{lstlisting}[language=TypeScript,caption={Erstellen eines Einkaufslisten‑Eintrags im Backend (\texttt{shoppingListsController.ts})}]
import { Request, Response } from "express";
import { getAuth } from "@clerk/express";
import * as service from "@/services/shoppingListsService";

export const createShoppingList = async (
req: Request,
res: Response
) => {
const { userId } = getAuth(req);
const shoppingList = await service.createShoppingList(
"Default List",
userId as string
);
res.json(shoppingList);
};
\end{lstlisting}

\section{Datenfluss im Anwendungsfall „Liste laden“}
\begin{enumerate}
\item \textbf{Initialisierung}: Beim Fokussieren der \texttt{ListScreen}-Komponente ruft \texttt{loadList()} die Methoden \texttt{api.checkShoppingListExists()} und \texttt{api.getShoppingListItems()} auf.
\item \textbf{HTTP‑Requests}: Axios sendet einen \texttt{GET /shoppinglist} bzw. \texttt{GET /shoppinglist/items} Request inklusive Bearer‑Token.
\item \textbf{Backend‑Authentifizierung}: Der Express‑Middleware \texttt{getAuth(req)} extrahiert das Clerk‑Token und identifiziert den Benutzer.
\item \textbf{Service‑Aufruf}: Die Controller‑Funktion holt die Liste aus der Datenbank über \texttt{shoppingListsService.getShoppingListItems(userId)}.
\item \textbf{Antwort an Frontend}: Der Controller serialisiert das Ergebnis als JSON, welches vom Frontend direkt in den State übernommen wird.
\end{enumerate}
