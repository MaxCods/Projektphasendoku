%%%%%%%%%%%%%%%%%%%%%%%%%%%%%%%%%%%%%
% INTEGRATION VON FRONTEND UND BACKEND
%%%%%%%%%%%%%%%%%%%%%%%%%%%%%%%%%%%%%
\chapter{Integration von Frontend und Backend}
\renewcommand{\authorinitials}{DH}

\label{chap:integration}

\section{Überblick der Client–Server-Kommunikation}
Die Kommunikation zwischen dem mobilen Frontend und dem Express‑Backend erfolgt über eine REST‑API, die im Frontend durch einen zentralen \texttt{ApiService} gekapselt wird. Dieser verwendet \texttt{Axios}, um HTTP‑Requests an das Backend zu senden und dabei automatisch das Authentifizierungs‑Token in die HTTP‑Header einzufügen.

\subsection{ApiService im Frontend}
Im Frontend ist die \texttt{ApiService}-Klasse in \texttt{api.ts} definiert. Diese Klasse stellt Methoden für alle CRUD‑Operationen (Create, Read, Update, Delete) der Einkaufslisten bereit. 

Ein typisches Beispiel ist die folgende Methode zur Erstellung einer neuen Einkaufsliste:

\begin{lstlisting}[language=TypeScript,caption={Definition der \texttt{createShoppingList}-Methode im \texttt{ApiService}}]
public async createShoppingList(): Promise<ShoppingListItem[]> {
  const response = await this.axiosInstance.post("/shoppinglist/");
  return response.data || [];
}
\end{lstlisting}

Hier wird über die Methode \texttt{axiosInstance.post()} ein \texttt{POST}-Request an den Endpunkt \texttt{/shoppinglist/} geschickt. Die Methode ist asynchron und gibt ein \texttt{Promise} mit den zurückgelieferten \texttt{ShoppingListItem}-Objekten zurück. Sollte der Server keine Daten zurücksenden, wird stattdessen ein leeres Array zurückgegeben. Die Nutzung einer dedizierten Instanz von Axios (\texttt{axiosInstance}) ermöglicht es, wiederkehrende Konfigurationen wie Basis-URLs oder Authentifizierungs‑Header zentral zu verwalten.

Die Nutzung dieser Methode im Frontend ist in der React‑Komponente \texttt{ListScreen.tsx} zu sehen:

\begin{lstlisting}[language=TypeScript,caption={Aufruf von \texttt{createShoppingList} im Frontend}]
const handleCreateShoppingList = async () => {
  setIsCreatingList(true);
  try {
    const apiResponse = await api.createShoppingList();
    setShoppingList(apiResponse);
    setHasShoppingList(true);
    Alert.alert("Success", "Shopping list created successfully!");
  } catch {
    Alert.alert("Error", "Could not create shopping list");
  } finally {
    setIsCreatingList(false);
  }
};
\end{lstlisting}

Diese Funktion ist als \texttt{async} deklariert, da sie auf den asynchronen API-Aufruf wartet. Zunächst wird der State \texttt{isCreatingList} auf \texttt{true} gesetzt, um dem Benutzer visuell anzuzeigen, dass eine Anfrage läuft (beispielsweise durch einen Ladeindikator). Anschließend wird \texttt{api.createShoppingList()} aufgerufen, um die neue Liste zu erstellen. Das Ergebnis (\texttt{apiResponse}) wird im State \texttt{shoppingList} gespeichert und mit \texttt{setHasShoppingList(true)} signalisiert, dass nun eine Liste existiert. Fehler werden über einen \texttt{catch}-Block abgefangen, der eine Fehlermeldung ausgibt, während \texttt{finally} sicherstellt, dass der Ladezustand am Ende wieder zurückgesetzt wird.

\section{Backend‑Implementierung der Endpunkte}
Das Backend stellt unter \texttt{/api/shoppinglist} eine REST‑Schnittstelle bereit. Diese ist in zwei Ebenen aufgeteilt: dem \emph{Router}, der die Routen definiert, und dem \emph{Controller}, der die Geschäftslogik ausführt.

\subsection{Router‑Definition}
Die Routing-Definition für die Einkaufslisten befindet sich in \texttt{shoppingLists.ts}:

\begin{lstlisting}[language=TypeScript,caption={Routing des Einkaufslisten‑Endpoints im Backend (\texttt{shoppingLists.ts})}]
import { Router } from "express";
import { createShoppingList, getShoppingListItems } from "@/controllers/shoppingListsController";
import { asyncHandler } from "@/utils/asyncHandler";

const router = Router();

// POST /api/shoppinglist/ -> createShoppingList
router.post("/", asyncHandler(createShoppingList));

// GET /api/shoppinglist/items -> getShoppingListItems
router.get("/items", asyncHandler(getShoppingListItems));

export default router;
\end{lstlisting}

Hier wird eine neue Router-Instanz von Express erstellt, die alle Routen im Kontext der Einkaufslisten verwaltet. Über \texttt{router.post("/")} wird ein Endpunkt zum Erstellen einer neuen Liste registriert, während \texttt{router.get("/items")} einen Endpunkt für das Abrufen von Listeneinträgen bereitstellt. Die Hilfsfunktion \texttt{asyncHandler} dient dazu, Fehler aus asynchronen Funktionen automatisch an die Express-Fehlerbehandlung weiterzuleiten, ohne dass explizite \texttt{try-catch}-Blöcke in den Routen notwendig sind.

\subsection{Controller‑Logik}
Die eigentliche Geschäftslogik, wie die Authentifizierung und die Verarbeitung der Anfrage, wird im Controller \texttt{shoppingListsController.ts} definiert:

\begin{lstlisting}[language=TypeScript,caption={Erstellen eines Einkaufslisten‑Eintrags im Backend (\texttt{shoppingListsController.ts})}]
import { Request, Response } from "express";
import { getAuth } from "@clerk/express";
import * as service from "@/services/shoppingListsService";

export const createShoppingList = async (
  req: Request,
  res: Response
) => {
  const { userId } = getAuth(req);
  const shoppingList = await service.createShoppingList(
    "Default List",
    userId as string
  );
  res.json(shoppingList);
};
\end{lstlisting}

Die Funktion \texttt{createShoppingList} extrahiert zunächst die \texttt{userId} des authentifizierten Benutzers aus der Anfrage, indem sie \texttt{getAuth(req)} aufruft. Anschließend wird der Service \texttt{shoppingListsService} genutzt, um für diesen Benutzer eine neue Liste mit dem Standardnamen „Default List“ anzulegen. Zum Schluss wird die erstellte Liste als JSON-Antwort an das Frontend zurückgesendet. Durch diese klare Trennung von Controller- und Service-Logik bleibt der Code modular und leichter wartbar.

\section{Datenfluss im Anwendungsfall „Liste laden“}
Der Ablauf zum Laden einer Liste gestaltet sich wie folgt:
\begin{enumerate}
  \item \textbf{Initialisierung}: Sobald die \texttt{ListScreen}-Komponente im Frontend in den Fokus rückt, ruft die Funktion \texttt{loadList()} die Methoden \texttt{api.checkShoppingListExists()} und \texttt{api.getShoppingListItems()} auf, um den aktuellen Zustand zu ermitteln.
  \item \textbf{HTTP‑Requests}: Über Axios werden zwei Anfragen gesendet: \texttt{GET /shoppinglist} und \texttt{GET /shoppinglist/items}. Beide Requests enthalten das Authentifizierungs-Token im \texttt{Authorization}-Header.
  \item \textbf{Backend‑Authentifizierung}: Auf der Serverseite wird das Token durch \texttt{getAuth(req)} validiert und der entsprechende Benutzer identifiziert.
  \item \textbf{Service‑Aufruf}: Der Controller greift auf \texttt{shoppingListsService.getShoppingListItems(userId)} zu, um die Daten aus der Datenbank abzurufen.
  \item \textbf{Antwort an das Frontend}: Die abgerufenen Daten werden als JSON an das Frontend zurückgesendet, wo sie direkt in den React-State übernommen werden.
\end{enumerate}
